\documentclass[12pt,journal,draftclsnofoot,onecolumn]{IEEEtran}

\usepackage{latexsym}
\usepackage{amssymb}
\usepackage{amsbsy}
\usepackage{amsmath}
\usepackage{multirow}
\usepackage{listings}
\usepackage{xcolor}

\lstset{ %numbers=left,
	%numberstyle= \tiny,
	basicstyle=\ttfamily\scriptsize,
	keywordstyle=\color{blue}\ttfamily,
	stringstyle=\color{red}\ttfamily,
	commentstyle=\color{green}\ttfamily,
	breaklines=true,
	showstringspaces=false,
	keepspaces=false,
	frame=shadowbox,
	extendedchars=false,
	tabsize=4,
	showspaces=false
	%rulesepcolor= \color{ red!20!green!20!blue!20}
}

\usepackage{longtable}
\usepackage{pifont}
\usepackage{times}
\usepackage{graphicx}
\usepackage{subfigure}
\pagestyle{plain}


\newcommand{\rmv}[1]{}
%\renewcommand{\baselinestretch}{1.6}
%\renewcommand{\thesection}{\Roman{section}.}
%\renewcommand{\thesection}{\arabic{section}}
%\renewcommand{\thesubsection}{\Alph{subsection}.}
%\renewcommand{\thesubsection}{\arabic{section}.\arabic{subsection}.}
%\renewcommand{\theequation}{\arabic{equation}}
%%\renewcommand{\theequation}{\arabic{section}.\arabic{equation}}
%\renewcommand{\thefigure}{\arabic{figure}}
%\renewcommand{\thetable}{\arabic{table}}
%\renewcommand{\thempfootnote}{\alph{footnote}}
%\renewcommand{\thempfootnote}{\fnsymbol{footnote}}

\newtheorem{theorem}{Theorem}
%\newtheorem{corollary}{Corollary}
\newtheorem{algorithm}{Algorithm}
%\newtheorem{definition}{Definition}
%\newtheorem{evaluation}{Evaluation}
\newtheorem{lemma}{Lemma}
\newtheorem{example}{Example}
\newtheorem{fact}{Fact}
%\newtheorem{property}{Property}

%\setcounter{page}{100}
\usepackage{url}
\usepackage{shorttoc}
\usepackage{color}

\begin{document}

%\mainmatter

\baselineskip=22 pt

\begin{titlepage}
\thispagestyle{empty}


\title{EE5414 Course Mini Project Report (BeagleBone Black)}

\author{\authorblockN{Wangchen DAI (53623708)}\\
\author{\authorblockN}{Jingwei HU (53656463)}\\
Instructor: Dr. L L CHENG\\
\authorblockA{Department  of Electronics Engineering\\
City University of Hong Kong}\\
\today}

\maketitle
\end{titlepage}

%\tableofcontents
\tableofcontents
\addcontentsline{toc}{section}{appendix}
\clearpage

\section{Introduction}\label{Intro}
An embedded system is a computer system with a specific function and embedded in a mechanical or electrical system. Due to the limitation of processing resources, lower power consumption, smaller size, lower cost, and simpler operating function are the main properties of embedded computers when compared with the general ones. Embedded systems are usually based on microcontrollers or microprocessors, such as MCU, ARM, DSP and FPGA. In this project, an ARM based circuit board, BeagleBone Black (BBB), is applied as the hardware development tool. Ubuntu Trusty 14.04, which is known as a version of open source Linux system, is loaded into the ARM chip as the software development platform. Using this specified embedded system, a series of functions are developed including: LED Test, Image/Video Capture, Web Server and SQL Server Establishment.


% % % % % % % % % % % % % % % % % % % % % % % % % % % % % % % % %
% % % % % % % % % % % % % % % % % % % % % % % % % % % % % % % % %
\section{Hardware Description and Implementation}\label{HdDes}

BBB is a low-cost, community-supported hardware device for embedded application development. BBB board mainly contains an ARM Cortex A8 series processor, a 512MB DDR3 RAM memory, an onboard 2GB MMC chip, with some other necessary peripherals. In this project, the BBB board is developed with following cables and devices: 
\begin{itemize}
\item A MiniUSB Cable, which can be used to connect the board with PC and served as power source (limited to 500mA), network, and serial port for data transmission;
\item An external DC power supply with 5V and a minimum of 1A current output;
\item A USB Hub to expand the onboard USB Host port from one to four;
\item A USB portable 802.11N wireless adapter;
\item A Logitech HD Pro C920 Web Camera, which is a USB webcam and provides full HD 1080p video recording in wide-screen at a maximum rate of 30 frames per second;
\item A mini SD card and its corresponding USB portable card reader.
\end{itemize}
The detailed and connection information of all hardware cables and devices is provided in both Fig.\ref{hw1}, for block diagram view, and Fig.\ref{hw2} for real view.
\begin{figure}[ht]
	\centering
	\includegraphics[width=4in]{./figs/hw1.jpg}
	\caption{Block Diagram of Hardware Description.}
	\label{hw1}
\end{figure}
\begin{figure}[ht]
	\centering
	\includegraphics[width=4in]{./figs/hw2.jpg}
	\caption{Real View of Hardware Description.}
	\label{hw2}
\end{figure}

Both of the figures indicate the hardware setup details. First, the BBB board is connected to a PC using the MiniUSB Cable. Via the cable, the BBB board could not only receives a 5V/500mA power supply provided by PC's USB port, but also be accessible from a host PC and share the Internet connection by setting up Windows ICS \cite{windowsICS}; In addition, users could operate Linux instructions through specific Windows-based software clients. Second, in case of unexpected system shutdown issue caused by current exceeding (current supply using MiniUSB Cable is set at a maximum of 500mA), an external power supply is used. Third, a USB Hub is connected to the onboard host USB port for expanding the port from one to four. Finally, the Logitech C920 webcam is connected to the USB Hub with its attached USB cable, together with a USB portable wireless adapter. Note that in this project, system could have two different IP addresses for Ethernet and wireless connection, respectively, also the BBB board could have an alternative way to access the Internet: getting access to a WiFi or Hotspot connection via the wireless adapter.


% % % % % % % % % % % % % % % % % % % % % % % % % % % % % % % % %
% % % % % % % % % % % % % % % % % % % % % % % % % % % % % % % % %
\section{Software Description and Implementation}\label{SfDes}
\begin{figure}[ht]
	\centering
	\includegraphics[width=5in]{./figs/sw1.jpg}
	\caption{Block Diagram of Software Description.}
	\label{sw1}
\end{figure}

In this project, the software implementation includes installation, configuration as well as application of embedded OS, Web server, PHP, and MySQL; setup and website-based control of onboard LEDs and the C920 Web Camera. Fig.\ref{sw1} provides a block diagram of software description.

	\subsection{Operating System Installation and Configuration}\label{Sys}
	In our implementation, we install Ubuntu Trusty 14.04 on BBB. This image uses the Ubuntu 14.04 core and thus  results  in an easy-to-install and stable Linux image that works well with the BBB boards.

First, OS binaries ``ubuntu-trusty-14.04-rootfs-3.14.4.1-bone-armhf.com.tar.xz" can be downloaded from \textcolor{blue}{\textit{\url{http://www.armhf.com/boards/beaglebone-black/}}}; After uncompressing this file, we can get the Ubuntu Trusty 14.04 image as shown in Fig.\ref{osimage}. Next, this image should be burn into a microSD card such that we can install Ubuntu 14.04 onto BBB. One way of doing this task is by utilising Win32DiskImager, which is a powerful tool automatically creating a OS boot disk and available at \textcolor{blue}{\textit{\url{http://sourceforge.net/projects/win32diskimager/files/latest/download}}}. In Fig.\ref{win32diskimager}, Win32DiskImager is starting to write Ubuntu.img to the microSD card, as presented by `device' F:. Please refer to the getting started page on the BeagleBone official website \cite{startBB} to find more details on the OS installation.

\begin{figure}[htb]
	\centering
	\includegraphics[width=4in]{./figs/osimage.PNG}
	\caption{Ubuntu Trusty 14.04 Image.}
	\label{osimage}
\end{figure}

\begin{figure}[htb]
	\centering
    \subfigure[MicroSD Card.]{
       \includegraphics[width=2in]{./figs/microsd.JPG}
       \label{microsd}
     }
    \subfigure[Win32DiskImager.]{
       \includegraphics[width=2.5in]{./figs/win32diskimager.PNG}
       \label{win32diskimager}
     }
     \caption{Creating A Ubuntu bootable image on microSD card.}
     \end{figure}

When the microSD Card is now ready to boot, we insert it into BBB board, hold down the USER/BOOT button (in our board, it is S3 button.) and apply power, either by the USB cable or 5V adapter. However, it is strongly recommended to plug the 5V adapter in all the time because power adapter can provide more stable voltage than USB cable and thus it is less likely to have unexpected problems caused by power cutoff or insufficient.

Keep holding down the button until  the bank of 4 LED's light up for a few seconds and then release the button.
It will take anywhere from 30-45 minutes to flash the image onto the on-board chip. Once it's done, the bank of 4 LED's to the right of the Ethernet will all stay lit up at the same time. We can then power down the BBB \cite{flashBB}. One important thing worth mentioning is that after finishing this step, we do not need microSD card for installation anymore --- you can plug it out and whenever you power the board again, Ubuntu is able to automatically boot from MMC, the inner storage component on chip.
	
\subsection{Wireless Adapter Configuration}\label{Wireless}
Once the OS is successfully set up on board, it becomes critical to have Internet access --- We need to install extra system utilities or download third-party libraries to support this project. So in this section, we detail how to set up a WiFi connection for BBB.

To begin with, we must connect the components needed into BBB: wireless network adapter, USB hub and miniUSB cable (See Fig.\ref{WiFi}). The wireless network adapter is the key for Internet connectivity. Besides this adapter, a web camera is also required by this project but we only have one USB port for them.  We therefore use a USB hub to install both components. We also need a miniUSB cable to connect BBB to the host PC such that we can remote control BBB through host PC's keyboard. Or to be more precise, we start a SSH connection to BBB through this cable.

\begin{figure}[htb]
	\centering
    \subfigure[Wireless Network Adapter.]{
       \includegraphics[width=2in]{./figs/wifiadapter.JPG}
       \label{wifiadapter.JPG}
     }
    \subfigure[Usb Hub.]{
       \includegraphics[width=2in]{./figs/usbhub.JPG}
       \label{usbhub.JPG}
     }
     \subfigure[MiniUSB Cable.]{
       \includegraphics[width=2in]{./figs/miniusbcable.JPG}
       \label{miniusbcable.JPG}
     }
     \caption{Essential components for WiFi connection.}
     \label{WiFi}
     \end{figure}


To establish SSH connection, we must log on BBB via SSH protocol and this could be done with the help of SSH tool. In fact, there are a variety of SSH GUI tools available for Windows. In this report, we take the Secure Shell Client available on \textcolor{blue}{\textit{\url{http://mepopedia.com/forum/file.php?793,file=1103,filename=SSHSecureShellClient-3.2.9.exe,download=1}}}. Because of the integration of $scp$ command, this tool allows us to transfer files between the host PC and the BBB freely.

When the wireless network adapter is plug into BBB and BBB itself is also connected to one USB port on our host PC, we open the Secure Shell Client and start a SSH connection from PC to BBB as shown in Fig.\ref{ssh}. Remember the default Ethernet IP address for BBB is 192.168.7.2 and the default user account is `ubuntu' with password "temppwd".
\begin{figure}[htb]
	\centering

    \subfigure{
       \includegraphics[width=2.5in]{./figs/ssh1.PNG}

     }
    \subfigure{
       \includegraphics[width=2.5in]{./figs/ssh2.PNG}

     }
     \caption{SSH log on via Secure Shell Client for Windows.}\label{ssh}
     \end{figure}

Finally, we configure the WiFi setup by adding the following commands in the network configuration file located at /ect/network/interfaces:
\begin{lstlisting}[language={bash}]
#WiFi Example
auth wlan0
iface wlan0 inet dhcp
#WPA Enterprise setup
wpa-driver wext
wpa-ssid Universities WiFi
wpa-ap-scan 1
wpa-proto RSN
wpa-pairwise CCMP TKIP
wpa-group CCMP TKIP
wpa-eap PEAP
wpa-key-mgmt WPA-EAP
wpa-identity *yourEID*@my.cityu.edu.hk
wpa-password *yourPASSWORD*
\end{lstlisting}
The idea of this setup is to connect to CityU Campus WiFi network --- "Universities WiFi" with the student EID and password.

To make the modifications into effect, we type in the following command from Secure Shell Client to restart the wirless adapter:
\begin{lstlisting}[language={bash}]
ifconfig wlan0 down && ifconfig wlan0 up
\end{lstlisting}
By typing following command, the function of WiFi connection can be verified:
\begin{lstlisting}[language={bash}]
ping www.google.com.hk
\end{lstlisting}
Fig.\ref{ping} shows that the network setup is correct with all 4 transmitted packets getting received.

\begin{figure}[htb]
	\centering
	\includegraphics[width=4in]{./figs/ping.PNG}
	\caption{A Successful Ping to Google.}
	\label{ping}
\end{figure}

\subsection{Web Server Setup}\label{Webser}
Apache together with PHP5 is used to establish the Web server. Apache can be installed on Ubuntu by typing in the terminal:
\begin{lstlisting}[language={bash}]
apt-get install apache2
\end{lstlisting}
Then, direct the browser to ``http://192.168.7.2" or ``http://*wireless IP address*", and the Apache placeholder page should appears: It Works! Note that the default document of placeholder page, or it can be called as homepage, can be found and edited at:
\begin{lstlisting}[language={bash}]
vi /var/www/index.php
\end{lstlisting}
	
In order to let Ubuntu  ``understand" PHP code, PHP5 and the Apache PHP5 modules must be installed. By installing both packages
\begin{lstlisting}[language={bash}]
apt-get install php5 libapache2-mod-php5
\end{lstlisting}
and restarting the Apache service
\begin{lstlisting}[language={bash}]
/etc/init.d/apache2 restart
\end{lstlisting}
PHP5 is configured and can be used for Web server developments. Please refer to \cite{Apache1} for detailed Web server configuration and PHP testing examples.

After setting up the Web server, we could build our own main webpage by replacing the default $index.php$ file mentioned previously, with a modified one. Here, a CityU style webpage is designed and replaces the original one as the project homepage, it can be accessed by typing either IP address of the board, see Fig.\ref{mainpage}. The template source code is provided by CityU of Hong Kong. All the subpages' source code is stored under the path $/var/www/php/$, which indicates that visiting any sub websites, you should direct the browser to ``http://192.168.7.2/php/*source\ file\ name*".
\begin{figure}[htb]
	\centering
	\includegraphics[width=4in]{./figs/mainpage.jpg}
	\caption{Project Homepage.}
	\label{mainpage}
\end{figure}
	
\subsection{LED Test}\label{Led}
The LED Test program is for testing the four LED lights located near the Ethernet port and the Reset Button. In general, each LED light has two configuration files: one controls the trigger mode:
\begin{lstlisting}[language={bash}]
echo none > /sys/class/leds/beaglebone:green:usr0/trigger
\end{lstlisting}
``none" for passive mode and it can be changed to ``heartbeat" for blink mode; the other file controls the On/Off switch of the light:
\begin{lstlisting}[language={bash}]
echo 1 > /sys/class/leds/beaglebone:green:usr0/brightness
\end{lstlisting}
1 for light on, and 0 for light off. Some of the LEDs are set to a default trigger mode which could display information to users about what is going on. Thus, the trigger mode must be set to passive mode before the state of light changes. And for blink mode, only the first command affects. Note that ``usr0" in the above paths can be changed to usr01, 2, and 3. Appendix \ref{AppendixLED} provides an example of LED\_0 control shell script. By modifying Appendix \ref{AppendixLED}, a shell script to control all four LEDs can be obtained. Such file is named as $ledctl.sh$ and an example of the all-four LEDs control command is given:
\begin{lstlisting}[language={bash}]
/var/www/LED/ledctl.sh on off on heartbeat
\end{lstlisting}
By calling this shell script in the $.php$ file,
\begin{lstlisting}[language={bash}]
<?php ...
$cmd = `/bin/bash ../LED/ledctl.sh $status0 $status1 $status2 $status3`;
... ?>
\end{lstlisting}
the website based LED test can be implemented. For example, Fig.\ref{LEDTest} indicates LED\_0 and 2 are switched on, LED\_1 is switched off, and LED\_3 is blink mode.
\begin{figure}[ht]
	\centering
	
	\subfigure[]{
		\includegraphics[width=4.5in]{./figs/LED1.jpg}\label{LEDTest1}		
	}
	\subfigure[]{
		\includegraphics[width=2in]{./figs/LED2.jpg}\label{LEDTest2}		
	}
	\subfigure[]{
		\includegraphics[width=2in]{./figs/LED3.jpg}\label{LEDTest3}		
	}
	\caption{Web based LED test.}\label{LEDTest}
\end{figure}

\subsection{Web Camera Configuration and Functioning}\label{Webcam}
The camera used in this project is Logitech HD Pro Webcam C920 \cite{logicam} (See Fig.\ref{logicam}), which is capable of delivering high-definition 1080p resolution and supports H.264 encoded compression to provide high quality video at low bit rates. As previously stated in section III-B, the camera is connected to BBB via the USB hub due to the limited number of USB ports on board.

In order to interact friendly with the web camber configuration, a webcam controller is implemented on our Apache web server. A webcamctl.php file is coded for basic functions like photographing, video capturing and face detection. The corresponding web page can be accessed by typing ``192.168.7.2/webcamctl.php" in the address bar of your web browser if BBB is connected to the host PC.  A snapshot of its web page is shown in Fig.\ref{webcamctl}.


\begin{figure}[htb]
	\centering

    \subfigure[Logitech HD Pro Webcam C920.]{
    \includegraphics[width=2.5in]{./figs/webcam4.PNG}\label{logicam}}
%    \subfigure[/var/www/php/webcamctl.php.]{
%       \includegraphics[width=3in]{./figs/webcam.PNG}
%     }
    \subfigure[Web Page for Webcam Controller.]{
       \includegraphics[width=3in]{./figs/webcam2.PNG}\label{webcamctl}

     }
     \caption{C920 Web Camera and Web Page.}\label{webcam}
     \end{figure}

It is suggested to use OpenCV library to achieve these functionalities because this library is cross-platform, which focuses mainly on real-time image processing and is free for use under the open source BSD license.
On top of this, OpenCV is pre-installed by default on our Ubuntu Trusty 14.04 operating system and thus it is not necessary to take pains learning how to immigrate OpenCV into BBB board. For photographing, we offer an option to enable face detection function or not. Besides, it automatically detects the edge of this picture by using image edge detection method;  For video capturing, we also provide with 4 options for the capture duration: 4 seconds, 8 seconds, 12 seconds and 16 seconds.  We present below in Fig.\ref{webcamdemo} the demo of this work by detecting the author himself.
\begin{figure}[htb]
	\centering
	\includegraphics[width=4in]{./figs/webcam3.PNG}
	\caption{Demonstrate for the webcam controller.}
	\label{webcamdemo}
\end{figure}

The details about the implementation of photographing and video capturing are introduced by Derek Molloy in his personal website \cite{molloy}. Please refer to \cite{molloy_videocapture,facedetection} for more information.
%
%
%The details about the implementation of photographing and video capturing is introduced by Derek Molloy in his personal website \cite{molloy}.
%Readers can refer to \textcolor{blue}{\textit{\url{http://derekmolloy.ie/beaglebone/beaglebone-video-capture-and-image-processing-on-embedded-linux-using-opencv/}}}
%for more information; As for the face detection functionality, readers can also refer to \textcolor{blue}{\textit{\url{http://stackoverflow.com/questions/20757147/opencv-and-cdetect-face-in-image}}} for help.



\subsection{SQL Server Setup and Application}\label{Sql}
First installing the MySQL packages , and getting MySQL support in PHP:
\begin{lstlisting}[language={bash}]
apt-get install mysql-server mysql-client php5-mysql
\end{lstlisting}
Then restart the Apache service:
\begin{lstlisting}[language={bash}]
/etc/init.d/apache2 restart
\end{lstlisting}
phpMyAdmin is a web interface through which the MySQL databases can be easily managed:
\begin{lstlisting}[language={bash}]
apt-get install phpmyadmin
\end{lstlisting}
A link must be established between phpMyAdmin and Apache:
\begin{lstlisting}[language={bash}]
ln -s /usr/share/phpmyadmin /var/www
\end{lstlisting}
Finally, by tying ``http://192.168.7.2/phpmyadmin" in the browser, and login with the username and password, the MySQL databases can be accessed with a graphical interface (See Fig.\ref{sql1}). Through which we could add, delete, set and insert data into databases.
\begin{figure}[ht]
	\centering
	\includegraphics[width=4.5in]{./figs/sql1.jpg}
	\caption{phpMyAdmin interface.}
	\label{sql1}
\end{figure}
\begin{figure}[ht]
	\centering
	\includegraphics[width=4in]{./figs/sql2.jpg}
	\caption{Inserted data in the tables of TempDB, MySQL.}
	\label{sql2}
\end{figure}

In this project, the main idea we use to insert captured videos and images into the database TempDB is: first, three $.cpp$ files are written, in which some mysql based functions are included in order to login to MySQL \cite{MySQL} and insert specified file (video, image, edge) into the right table of database TempDB (see Appendix \ref{AppendixSQL} as an example); then compile this files using g++, taking $video.cpp$ as an example:

\begin{lstlisting}[language={bash}]
g++ video.cpp -o video -I/usr/include/mysql mysqlclient
\end{lstlisting}
Finally, call all these three compiled files in $webcamctl.php$, so that the captured videos and images can be stored into the specified tables in TempDB with a corresponding timestamp. (See Fig.\ref{sql2})

% % % % % % % % % % % % % % % % % % % % % % % % % % % % % % % % %
% % % % % % % % % % % % % % % % % % % % % % % % % % % % % % % % %	
\section{Conclusions}\label{Con}
In this project, we use BBB and embedded ARM-Linux as the hardware and software development tools, respectively. Implementing a series of functions including: Web server and SQL server setup and configuration; webpage based LED Test, Image/Video Capture and face detection;  MySQL based data storage functionality.

% % % % % % % % % % % % % % % % % % % % % % % % % % % % % % % % %
% % % % % % % % % % % % % % % % % % % % % % % % % % % % % % % % %
\section*{Acknowledgment}
\addcontentsline{toc}{section}{Acknowledgment}
We would like to thank our colleague Sean Yang  for his kindly help with the WiFi setup on BBB board.

\bibliographystyle{plain}
\bibliography{bibfile}
\clearpage

\appendix
    \subsection{An example of LED\_0 control shell script file}\label{AppendixLED}
    \begin{lstlisting}[language={bash}]
    #!/bin/bash
    case "$1" in
    'on')
    echo "LED_0 is turned on"
    sudo bash -c "echo none > /sys/class/leds/beaglebone:green:usr0/trigger"
    sudo bash -c "echo 1 >    /sys/class/leds/beaglebone:green:usr0/brightness"
    ;;
    'off')
    echo "LED_0 is turned off"
    sudo bash -c "echo none > /sys/class/leds/beaglebone:green:usr0/trigger"
    sudo bash -c "echo 0 >    /sys/class/leds/beaglebone:green:usr0/brightness"
    ;;
    'heartbeat')
    echo "LED_0 is blinking"
    sudo bash -c "echo heartbeat > /sys/class/leds/beaglebone:green:usr0/trigger"
    ;;
    esac
    \end{lstlisting}
    
	\subsection{An example of how to insert a file into a SQL database}\label{AppendixSQL}	
	\begin{lstlisting}[language={c}]
	#include <mysql.h>
	#include <stdlib.h>
	#include <stdio.h>
	#include <string.h>
	
	void error_exit(MYSQL *conn){
	fprintf(stderr, "%s\n", mysql_error(conn));
	if (conn != NULL){
	mysql_close(conn);
	}
	exit(1);
	}
	
	int main(){
	MYSQL *conn = mysql_init(NULL);
	if(conn == NULL){
	error_exit(conn);
	}
	
	if (mysql_real_connect(conn, "localhost", "root", "temppwd", "TempDB", 0, NULL, 0) == NULL)	{
	error_exit(conn);
	}
	
	if (mysql_query(conn, "CREATE TABLE IF NOT EXISTS Capture_Video(id INT PRIMARY KEY AUTO_INCREMENT, TimeStamp DATETIME, movie LONGBLOB)")){
	error_exit(conn);
	}
	
	mysql_query(conn, "INSERT INTO Capture_Video(TimeStamp, movie) VALUES(NOW(), LOAD_FILE('/var/www/WebCam/output.mp4'))");
	mysql_close(conn);
	return 0;
	}
	\end{lstlisting}

\end{document}


